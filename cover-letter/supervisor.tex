\documentclass[letterpaper,10pt]{article}

\usepackage{cite}
\usepackage{enumerate}
\usepackage{url}

\hyphenation{Ge-nome Ge-nomes through-put re-comb-in-ation}


% Define FH official colors
\usepackage[dvinames]{xcolor}
% \definecolor{FHgray}{HTML}{5e6a71} % a gray that seems a little light
\definecolor{FHgray}{HTML}{213861} % actually the blue from the logo

% Measurements are taken directly from the guide
\usepackage[top=2in,left=1.5in,bottom=0.5in,right=0.625in]{geometry}
\usepackage{graphicx}
\usepackage[colorlinks=false,
            pdfborder={0 0 0},
            ]{hyperref}
\usepackage[absolute]{textpos}
\usepackage{ifthen}
\usepackage{soul}

% \usepackage[sc,osf]{mathpazo}
\linespread{1.05}

\usepackage{baskervald} % http://www.ctan.org/pkg/baskervaldadf (see http://www.tug.org/fonts/fontinstall.html)
%\usepackage{classico} % http://www.ctan.org/pkg/classico
\usepackage{gillius2}

% Remove paragraph indentation
\parindent0pt
\setlength{\parskip}{0.8\baselineskip}
% \raggedright
\pagestyle{empty}
% Ensure consistency in the footer
\urlstyle{sf}


\providecommand\FHfromname{Frederick ``Erick'' A Matsen IV}
\providecommand\FHfromtitle{Associate Member}
\providecommand\FHfromdegree{Ph.D.}
\providecommand\FHfromdept{Computational Biology Program}
\providecommand\FHfromaddress{Fred Hutchinson Cancer Research Center \hfill 1100 Fairview Ave N, Mail stop M1-B514, Seattle WA 98109-1024}
\providecommand\FHfromtel{+1 206 667 7318}
\providecommand\FHfromfax{+1 206 667 2437}
\providecommand\FHfromemail{\href{mailto:matsen@fredhutch.org}{matsen@fredhutch.org}}
\providecommand\FHfromweb{\url{http://matsen.fredhutch.org/}}
\providecommand\FHtoname{Dr.~Wiseman}
\providecommand\FHtoaddress{Department of Wisemen\\
                            Wise State University \\
                            Wisetown, Wise\ \ 12345-6789\\
                            USA}
\providecommand\FHdate{\today}
\providecommand\FHopening{Dear \FHtoname,}
\providecommand\FHclosing{Sincerely}
% Update this and the next line to the correct path
\providecommand\FHsignaturefile{sig}
\providecommand\FHlogofile{fredhutch-logo}
\providecommand\FHenclosure{}

\usepackage{fancyhdr}
\pagestyle{fancy}

\renewcommand{\footrulewidth}{0pt}
\fancyfoot{}
\fancyfoot[L]{%
    {\footnotesize\color{FHgray}\sffamily
    \FHfromweb \hfill \FHfromemail\ \ \FHfromtel\\[-0.1\baselineskip]
    \FHfromaddress}\color{black}}

\fancyhead{}
\fancyhead[L]{%
    \begin{textblock*}{2in}[0.3066,0.39](1.8in,0.83in) % (x, y) pos
        \includegraphics[width=2in]{\FHlogofile}
    \end{textblock*}
    \begin{textblock*}{6.375in}(1.25in,0.7in)   % 6.375=8.5 - 1.5 - 0.625
        \sffamily
        \hfill \color{FHgray} \FHfromdept
        \\ \hfill \FHfromname, \FHfromdegree
    \end{textblock*}
\renewcommand{\headrulewidth}{0pt}
}

\AtBeginDocument{
    % Text lines should be less than 6in long
    \newgeometry{top=1.5in,left=1.25in,bottom=1.2in,right=1.25in}

    \bigskip
    \FHdate
    \bigskip

    \FHtoname\ifthenelse{\equal{\FHtoname}{}}{}{\\}
    \FHtoaddress
    \bigskip

    \FHopening\par
    }

\AtEndDocument{
    \par\vspace{2ex}
    \FHclosing,
    \par\vspace{2ex}
    \ifthenelse{\equal{\FHsignaturefile}{}}{\bigskip\bigskip}{\includegraphics[width=2.0in]{\FHsignaturefile}\\[-0.9\baselineskip]}

    \FHfromname, \FHfromdegree\\
    \FHfromtitle, \FHfromdept\\
    Affiliate Associate Professor, University of Washington Depts of Statistics and Genome Sciences \\
    HHMI/Simons Faculty Scholar
    \\
    \FHenclosure
}



\renewcommand\FHtoname{}
\renewcommand\FHtoaddress{}

\renewcommand\FHopening{Dear MBE Editorial Board,}

\begin{document}

Please consider our manuscript entitled ``Joint maximum-likelihood of phylogenies and ancestral states is not consistent'' for publication in \textit{Molecular Biology and Evolution}.
In this paper, my co-author David Shaw and I investigate the statistical properties of joint maximum-likelihood estimation of trees and ancestral sequences.
We were motivated to do so by recent work from Richard Neher's group (\url{http://dx.doi.org/10.1093/ve/vex042}) that employs such joint optimization as an efficient alternative to the standard Felsenstein marginalization.

We show that such joint inference is not consistent, both for topology inference and branch length optimization.
Indeed, there are sizeable regions of parameter space in which this joint inference gives the wrong topology and branch lengths in the limit of large data.
This is not a case of long branch attraction.
Indeed, the inconsistency in Felsenstein 1978 requires a combination of long and short branches, whereas our inconsistency comes from all branches being long.

Although the proofs in our paper require detailed examination of likelihood functions, we have written the paper to be broadly accessible.
The paper has five illustrations showing various regions of inconsistency, and provides intuitive summaries of the results in the main text.
A complete set of proofs appears in the appendix.

We feel that it is important for this paper to get a critical review by a mathematically-oriented phylogenetics expert.
We suggest:
\begin{itemize}
\setlength\itemsep{1.4pt}
\item Elizabeth Allman \texttt{esallman@alaska.edu}
\item Laura Kubatko \texttt{lkubatko@stat.osu.edu}
\item Mike Steel \texttt{mike.steel@canterbury.ac.nz} (although Mike is somewhat conflicted as my postdoc advisor, he would do an excellent job reviewing this paper)
\item Ed Susko \texttt{susko@mathstat.dal.ca}
\item Richard Neher \texttt{richard.neher@unibas.ch}
\end{itemize}

Thank you for your consideration.
\end{document}
