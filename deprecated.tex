\section{Temporarily unused text I may want later to show more general results on the likelihood estimates}

$$
a = p(0100) + p(1000) + p(0011) + p(1001) = (4-2x^2-2y^2)/8,
$$
$$
b = p(1000) + p(0010) + p(0011) + p(1001) = (4-2x^2-2y^2)/8,
$$
$$
c = p(0001) + p(0010) + p(0100) + p(1000) = (4-4x^2y^2)/8
$$
and
$$
d = p(0101) + p(1001) = (2-4xy^2+2x^2y^2)/8.
$$
Some scratch work:
$$
a = b = (4-2x^2-2y^2)/8,
$$
$$
c+2d = 1-xy^2, \ 2-c-2d = 1+xy^2
$$
$$
H_{fe}(x,y; 1, 0, 1-c-2d) \propto (2-4xy^2+2x^2y^2)\log(1-2xy^2+x^2y^4) + (2+4xy^2+2x^2y^2)\log(1+2xy^2+x^2y^4)+(4-4x^2y^2)\log(1-x^2y^4)
$$
$$
H_{fa}(x,y; x, y, y) \propto (2-4xy^2+2x^2y^2)\log(1-2xy^2+x^2y^4) + (2+4xy^2+2x^2y^2)\log(1+2xy^2+x^2y^4)+(4-4x^2y^2)\log(1-x^2y^4)
$$
$$
\hat{L}_{fe} \propto 2\log(2) + (1+xy^2)\log(1+xy^2) + (1-xy^2)\log(1-xy^2)
$$
$$
\hat{L}_{fa} \propto \log(2) + (4-2a)\log(2-a)+2a\log(a).
$$

\subsubsection{Refinements}

It will be an undertaking, but if we want to approach this using the refinements we might get something pretty nifty.
Everything will be quite messy, but really the entire approach boils down to a series of nested inequalities in two variables.
We may need a program that can reduce symbolic equations, but this will at least get us somewhere that isn't a lot of loose approximations---initial results using numerical optimization show the region in~\ref{fig:inconsistency-farris} is conservative, and very much so.
See Fig.~\ref{fig:numerical-inconsistency-farris}.

Perhaps something along the lines of a more general results would be interesting?
Like, could we calculate the measure/area of inconsistency using cooler techniques?
Could we do this for a general topology?
Would that even be interesting since we've already found a counterexample?

It also seems that, in the $\{x, y\}=\{.25, .25\}$ example, the lower bound is tight while the upper bound is looser.
This surprises me since we replaced a lot in the likelihood, but this might indicate that the $H$ term is actually the one that varies more than the $L$ term.
I guess this is intuitive: the difference between the lower and upper bound on $L$ is $\log2\approx0.69$.
Maybe getting a tighter upper bound will be useful here?

\begin{figure}
\centering
\includegraphics[width=.9\textwidth]{ineqs-num-far-gen}
\caption{Regions of inconsistency---numerically obtained}
\label{fig:numerical-inconsistency-farris}
\end{figure}

\subsubsection{Tighter upper bound}

Call
$$
e = p(0001) + p(0100) = \frac{1+x^2-y^2-x^2y^2}{4},
$$
$$
f = p(0010) + p(1000) = \frac{1-x^2+y^2-x^2y^2}{4}.
$$
Then some algebra and standard $\log$ bounds can be used to get an upper bound for the full Farris likelihood $L^{(i)}$ as
$$
(2-a+e)\log(2-\frac{2a}{2+e}) + a\log\frac{2a}{2+e} + (2-b+f)\log(2-\frac{2b}{2+f}) + a\log\frac{2b}{2+f}.
$$
In our example of $\{x,y\}=\{.25,.25\}$ this bound is $a$ compared to $-4.25313560433$ before.


\subsubsection{A counterexample in branch lengths}

Let $(\tau^*, t^*)=(\tau_1, \{x,y,x\})$.
We claim there exist $\{x,y\}$ and $\{x, \tilde{y}\}$ such that
$$
L'(\tau^*, t^*; \tau^*, \{x, \tilde{y}\}) > L'(\tau^*, t^*; \tau^*, t^*).
$$
The model is given by values in the Table.
Set $x=0$ to obtain
$$
L'(\tau^*, t^*; \tau^*, t^*) \propto 8\log(1+y)
$$
and
$$
L'(\tau^*, t^*; \tau^*, \{0, \tilde{y}\}) \propto 8\log(1+\tilde{y}).
$$
Therefore, if $\tilde{y} > y$ we have an inconsistency.

\begin{table}
\centering
\begin{tabular}{|l|l|l|}
    \hline
$s_j$   &$P(s_j|\tau_1,t)$&$\max_{z_j} \ P(s_j,z_j|\tau_1,t)$\\
    \hline
0000&$1+2x^2+4x^2y+x^4$&$(1+x)^4(1+y)$\\
0001&$1-x^4$&$(1+x)^3(1+y)(1-x)$\\
0010&$1-x^4$&$(1+x)^3(1+y)(1-x)$\\
0100&$1-x^4$&$(1+x)^3(1+y)(1-x)$\\
1000&$1-x^4$&$(1+x)^3(1+y)(1-x)$\\
0011&$1+2x^2-4x^2y+x^4$&max$\{(1+x)^2(1+y)(1-x)^2,(1+x)^4(1-y)\}$\\
0101&$1-2x^2+x^4$&$(1+x)^2(1+y)(1-x)^2$\\
1001&$1-2x^2+x^4$&$(1+x)^2(1+y)(1-x)^2$\\
    \hline
\end{tabular}    
\caption{Counterexample 1.}
\label{tab:sitepatprob_case1}
\end{table}

\subsubsection{A counterexample in topologies}

Let $(\tau^*, t^*)=(\tau_1, \{x,y,y\})$.
We claim there exist $\{x,y\}$ such that
$$
L'(\tau^*, t^*; \tau_2, t^*) > L'(\tau^*, t^*; \tau^*, t^*).
$$
Set $y=0$ to obtain parameters given in Table.
We have
$$
L'(\tau^*, t^*; \tau^*, t^*) \propto 4(1+x^2)\log(1+x^2)+4(1-x^2)\log(1-x^2)+4(1+x^2)\log((1+x)^2)+4(1-x^2)\log((1+x)(1-x))
$$
and
$$
L'(\tau^*, t^*; \tau_2, t^*) \propto 3(1+x^2)\log((1+x)^2)+3(1-x^2)\log((1+x)^2)+(1+x^2)\log((1+x)(1-x))+(1-x^2)\log((1+x)(1-x)).
$$
The difference being greater than zero, i.e.,
$$
L'(\tau^*, t^*; \tau_2, t^*) - L'(\tau^*, t^*; \tau^*, t^*) > 0,
$$
simplifies to
$$
4x^2(\log(1-x^2)-\log(1+x^2)+\log(1-x)-\log(1+x)) - 2(2\log(1+x^2)+2\log(1-x^2)-\log(1+x)+\log(1-x)) > 0.
$$
Let $0 < x < 1/\sqrt{2}$ so that the above inequality becomes
$$
-\log(1-x^2)-3\log(1+x^2) > 0
$$
or
$$
(1+x^2)^3(1-x^2) < 1.
$$
Applying $x<1/\sqrt{2}$ shows if
$$
x < \sqrt{-1+2^{1/3}} \approx 0.5098
$$
we have an inconsistency.

\begin{table}
\centering
\begin{tabular}{|l|l|l||l|l|}
    \hline
$s_j$   &$P(s_j|\tau_1,t)$&$\max_{z_j} \ P(s_j,z_j|\tau_1,t)$&$P(s_j|\tau_2,t)$&$\max_{z_j} \ P(s_j,z_j|\tau_2,t)$\\
    \hline
0000&    $1+x^2$ & $(1+x)^2$ & 1 & $(1+x)^2$\\
0001&    $1+x^2$ & $(1+x)^2$ & 1 & $(1+x)^2$\\
0010&    $1+x^2$ & $(1+x)^2$ & 1 & $(1+x)^2$\\
0100&    $1-x^2$ & $(1+x)(1-x)$ & 1 & $(1+x)^2$\\
1000&    $1-x^2$ & $(1+x)(1-x)$ & 1 & $(1+x)^2$\\
0011&    $1+x^2$ & $(1+x)^2$ & 1 & $(1+x)(1-x)$\\
0101&    $1-x^2$ & $(1+x)(1-x)$ & 1 & $(1+x)^2$\\
1001&    $1-x^2$ & $(1+x)(1-x)$ & 1 & $(1+x)(1-x)$\\
    \hline
\end{tabular}    
\caption{Counterexample 2.}
\label{tab:sitepatprob_case2}
\end{table}

\subsubsection{More fleshed out counterexamples}

The above counterexamples are highly pathological, but they show there exist whole spaces of parameters yielding inconsistencies.
It seems as long as $x$ and $y$ are ``small enough'' we see inconsistency.

\subsection{start}

We can calculate the probability of a particular site pattern under this topology from the probability that all character states are equal as
$$
P(0000|\tau^*,t) = \frac{1}{8}(1+x^2+z^2+4xyz+x^2z^2)
$$
where the probability of any site pattern will be the same as above with the corresponding branch length parameter replaced with its negative.
For example, if we have the pattern $0001$, since the last branch has length $z$ we replace it with $-z$ and obtain
$$
P(0001|\tau^*,t) = \frac{1}{8}(1+x^2-z^2-x^2z^2).
$$
All possible site pattern probabilities are given in Table~\ref{tab:sitepatprob}.
All likelihoods and their maximal values are given in Table~\ref{tab:likelihoods}.

It can be shown (I'll do it later) that for our $t^*$ the right-hand term above is $-16\log 8$.
Moreover, for $t=\{x,y,z\}$
\begin{equation}
\sum_{s} \log P(s | \tau^*, t) + \log P(\xi | s, \tau^*, t) = -16\log 8 + f(x,y,z) + g(x,y,z)
\end{equation}
where
\begin{align}
f(x,y,z) &= 4\log(1+x)+4\log(1-x)+4\log(1+z)+4\log(1-z)+\\
& 3\log(1+x^2)+3\log(1+z^2)+\\
& \log((1+x^2)(1+z^2)+4xyz) + \log((1+x^2)(1+z^2)-4xyz).
\end{align}
and
\begin{align}
g(x,y,z) &= 10\log(1+x)+10\log(1+z)+4\log(1-x)+4\log(1-z)+7\log(1+y)+\\
& \left\{ 
  \begin{array}{l}
  2\log(1+x)+2\log(1-z)+\log(1+y)\\
  2\log(1+x)+2\log(1+z)+\log(1-y)\\
  2\log(1-x)+2\log(1-z)+\log(1+y)\\
  \end{array} \right.
\end{align}
whose component will be chosen based on whichever yields the maximum value for $\log L''$.

\subsection{Two fixed topologies}

\subsection{Special case: $x=z=0$}

Consider now setting all data-generating branch length parameters $t^*$ to zero; this yields equal probabilities of $1/8$ for each site pattern.
By the inequality of Gibbs we have
$$
\sum_{s} P(s|\tau^*,t^*)\log P(s|\tau^*,t^*) \geq \sum_{s} P(s|\tau^*,t^*)\log P(s|\tau^*,t)
$$
summing over all site patterns, which is equivalent to
$$
\sum_{s} \log P(s|\tau^*,t^*) \geq \sum_{s} \log P(s|\tau^*,t).
$$
We must show, then, that there exist $t\neq t^*$ such that
$$
\max_{\xi_s} \ \sum_{s} \log P(s | \tau^*, t) + \log P(\xi_s | s, \tau^*, t) > \max_{\xi_s} \sum_{s} \log P(s | \tau^*, t^*) + \log P(\xi_s | s, \tau^*, t^*).
$$
Assume for $t$ that $x=z=0$ but $y\neq 0$.
Clearly site pattern probabilities are still all equal, and the first summand of each side of the inequality can be ignored in this case.
Call
$$
f(x,y,z) = \sum_{s} \log P(\xi_s | s, \tau^*, t=\{x,y,z\}).
$$
For branch lengths $\{x,y,z\}$, by considering all eight cases (see Table~\ref{tab:likelihoods}) of site patterns we can show that in terms of branch length \emph{probabilities}%
	%
    \footnote{The function $f$ has inverse $f^{-1}(u)=(1-u)/2$ so that the probability of a change along a branch of length $x$ maps to $(1-x)/2$ and the probability of no change maps to $(1+x)/2$.}
    %
\begin{align}
f(x,y,z) \propto & \ 10\log(1+x)+10\log(1+z)+4\log(1-x)+4\log(1-z)+7\log(1+y)+\\
& \left\{ 
  \begin{array}{l}
  2\log(1+x)+2\log(1-z)+\log(1+y)\\
  2\log(1+x)+2\log(1+z)+\log(1-y)\\
  2\log(1-x)+2\log(1-z)+\log(1+y)\\
  \end{array} \right.
\end{align}
whose last summand will be chosen based on whichever yields the maximum value and where the proportionality is due to requiring branch length parameters $\{x,y,z\}$ to be multiplied by $1/2$ to map them to probabilities.
We see that
$$
f(x,y,z)|_{x=z=0} = 8\log(1+y)
$$
and thus require the constraint
$$
8\log(1+y) > 0
$$
or that $y$ is any value bounded away from zero, and will obtain a maximum at $y=1$ or $p_y=0$.

\subsection{Special case: $x=z\neq 0$}


Fixing $x=z=0$ is a little too pathological---anything can happen when we're flipping coins to determine terminal characters.
Let's investigate a relaxed yet still special case where branches to the tips are all equal but allowed to be nonzero.
So we set $\tau^*=\tau$, $t^*=\{x,x,y\}$ and $t=\{x,x,\tilde{y}\}$ where the same terminal branches are used for both parameter sets with only the internal branch different.
Removing terms involving only $x$ and redundant $y$ shows that we need only concern ourselves with the objective functions
\begin{align*}
L_1(u) =& 8\log(1+u) + \\
  &(1+2x^2+4x^2y+x^4)\log(1+2x^2+4x^2u+x^4) + \\
  &(1+2x^2-4x^2y+x^4)\log(1+2x^2-4x^2u+x^4)
\end{align*}
if $(1+u)(1-x)^2 > (1-u)(1+x)^2$---or, more simply, if $u > 2x / (1+x^2)$---and
\begin{align*}
L_2(u) =& (7-2x^2+4x^2y-x^4)\log(1+u) + \\
  &(1+2x^2-4x^2y+x^4)\log(1-u) + \\
  &(1+2x^2+4x^2y+x^4)\log(1+2x^2+4x^2u+x^4) + \\
  &(1+2x^2-4x^2y+x^4)\log(1+2x^2-4x^2u+x^4)
\end{align*}
otherwise.
We can rule out the case where $\tilde{y} < y$ in $L_1$ by the inequality of Gibbs and using a similar argument along with the inequality
$$
1+2x^2-4x^2y+x^4 \le 7-2x^2+4x^2y-x^4
$$
implies we can rule out $\tilde{y} < y$ in $L_2$ as well (actually I'm not sure if I believe this anymore; the case of $L_1$ definitely holds, though).
This leaves us with four sets of inequalities where an inconsistency will arise,
$$
\left\{\tilde{y} > y > \frac{2x}{1+x^2}, \ L_1(\tilde{y}) > L_1(y)\right\}, 
$$
$$
\left\{\tilde{y} > \frac{2x}{1+x^2} > y, \ L_1(\tilde{y}) > L_2(y)\right\}, 
$$
$$
\left\{\frac{2x}{1+x^2} > \tilde{y} > y, \ L_2(\tilde{y}) > L_2(y)\right\},
$$
and
$$
\left\{\frac{2x}{1+x^2} > y > \tilde{y}, \ L_2(\tilde{y}) > L_2(y)\right\}.
$$
Simplifying shows
$$
L_2(u) = L_1(u) + (1+2x^2-4x^2y+x^4)\log\frac{1-u}{1+u} \le L_1(u).
$$
Can we use this fact?
I believe in my heart of hearts, perhaps foolishly, that the first set of inequalities holds categorically.
The fact above will then imply the second set of inequalities must also hold.
The tricky bits are the last two sets of inequalities.

If you'll allow me to indulge in a bit of cheating, I provide a few plots showing---for fixed $x$---the regions where inconsistencies arise.
It looks like there is some regularity that we must be able to exploit somehow, if only to show that for a fixed $x$ there are inequalities in $y$ and $\tilde{y}$ governing consistency.

%\begin{figure}
%\centering
%\includegraphics[width=.3\textwidth]{ineqs-x-0_25}\hfill
%\includegraphics[width=.3\textwidth]{ineqs-x-0_75}\hfill
%\includegraphics[width=.3\textwidth]{ineqs-x-0_9}
%\caption{Various regions of inconsistency for fixed $x$}
%\end{figure}

Say $\tau_1$ is the ``Felsenstein zone'' tree and $\tau_2$ is the ``Farris zone'' tree, but both trees have a different value for its middle, connecting branch.
%EM widdershins!! This still depends on how we draw them, no?
The tips are labeled widdershins proceeding from the top-left tip.
\begin{verbatim}
Felsenstein:
1\      /3
   >--< 
2/      \4

Farris:
1\ /3
  | 
2/ \4
\end{verbatim}
Our branch length parameters $t=\{x,y,z\}$ will consist of transformed probabilities where $y$ corresponds to this middle branch.
For $p_x\in[0,1/2]$ defined as
$$p_x = Pr(\mbox{change occurred along branch})$$
then $x = 1-2p_x$.
The parameter $x$ takes values in $[0,1]$ and will appear in likelihood calculations as either $(1+x)$ or $(1-x)$.
Properties for $y$ and $z$ are identical.
Assume for the moment that $\tau^*=\tau_1$ and $t^*=\{x,y,z\}$.

\subsubsection{Example site pattern: \texttt{0101}}

It can be shown that the only $A$ for which there is more than one partition for $\tau_2$ is for the site pattern \texttt{0101}.
In this case
$$A_1(\tau_1) = \{(1+x)^2(1-z)^2 = (1-x)^2(1+z)^2\},$$
$$A_2(\tau_1) = \{(1+x)(1+y)(1-z) = (1-x)(1-y)(1+z)\},$$
$$A_3(\tau_1) = \{(1-x)(1+y)(1+z) = (1+x)(1-y)(1-z)\}$$
and
$$A_1(\tau_2) = \{(1-z)^2(1+y) = (1+z)^2(1-y)\},$$
$$A_2(\tau_2) = \{(1-z)^2(1+x)^2 = (1+z)^2(1-x)^2\},$$
$$A_3(\tau_2) = \{(1+x)^2(1-y) = (1-x)^2(1+y)\}.$$

Now, the probability of generating this site pattern under $\tau^*$ is given as
$$
Pr(\texttt{0101} | \tau^*, t^*) = \frac{1}{8}(1-2xz+y(x^2+z^2-2xz)+x^2z^2).
$$

\subsubsection{Example site pattern: \texttt{0000}}

In the case of site pattern \texttt{0000} the refinements $A$ for both topologies are single-element partitions.
In other words, no configuration of branch lengths will result in an ancestral state other than \texttt{00}.
Here we have the probability
$$
Pr(\texttt{0000} | \tau^*, t^*) = \frac{1}{8}(1+2xz+y(x^2+z^2+2xz)+x^2z^2)
$$
and the maximum value for $Pr(\xi | \texttt{0000}, \tau, t^*)$ for both $\tau_1$ and $\tau_2$ is
$$
Pr(\xi | \texttt{0000}, \tau, t^*) = (1+x)^2(1+z)^2(1+y).
$$

\subsubsection{All refinements with multi-element partitions}

(show calculations for all refinements that have multiple partitions)

\subsection{Numerical sanity check}

Let's suppose site pattern \texttt{0101} is the only one that drives some disconcert between the generating tree and the maximum likelihood tree.
For a number of combinations of $\{x,y,z\}$ we need to calculate first
$$
Pr(\texttt{0101} | \tau^*, t^*) = \frac{1}{8}(1-2xz+y(x^2+z^2-2xz)+x^2z^2)
$$
and the maximum value over each $A_i(\tau)$ of $Pr(\xi | \texttt{0101}, \tau, t^*)$ for $\tau\in\{\tau_1, \tau_2\}$.
In other words, say we have the triplet $\{x,y,z\}=\{0.8,0.5,0.8\}$.
Then we have
$$
\max_{\xi} Pr(\xi | \texttt{0101}, \tau_1, t^*) = 0.0243, \ \max_{\xi} Pr(\xi | \texttt{0101}, \tau_2, t^*) = 0.6561, \ Pr(\texttt{0101} | \tau^*, t^*) = 0.0162.
$$
This doesn't yet tell us anything, as the probabilities for other site patterns may overwhelm this particular one, but gives a general flavor of how these computations will be performed.
Moreover, a different triplet of $\{x,y,z\}$ could obtain a different maximum value for $Pr(\xi | s, \tau, t^*)$.
This can be seen with the pattern \texttt{0011} where
$$
\max_{\xi} Pr(\xi | \texttt{0011}, \tau_1, t^*) = 0.6561, \ \max_{\xi} Pr(\xi | \texttt{0011}, \tau_2, t^*) = 0.0243, \ Pr(\texttt{0011} | \tau^*, t^*) = 0.1762
$$
meaning that if all else were equal---i.e., $\max_{\xi} Pr(\xi | s, \tau, t^*)$ were the same for $\tau_1$ and $\tau_2$ for all $s$---and that the maximum value actually occurs at this particular triplet $\{x,y,z\}$ then
$$
L(\tau_1) = 0.928 > 0.119 = L(\tau_2)
$$
and we have no contradiction.
That's a lot of ``if'' statements, so let's investigate in a little more generality.

\subsection{Differing branch lengths}

First, we do not wish to assume $\{x,y,z\}$ are the same for both trees.

\subsubsection{General case}

The generating probability $Pr(s | \tau^*, t^*)$ has a specific form given a Hadamard transformation of the branch length parameters, while the term proportional to $Pr(\xi | s, \tau, t^*)$ relies on taking the maximum over a function with multiple components.
All of this is easy to write into a computer program, and we can numerically find the maximum value over $\{\theta_1, \ldots, \theta_5\}\in[0,1]$ of each $L(\tau_1)$ and $L(\tau_2)$.
If $L(\tau_1) < L(\tau_2)$ we see that jointly maximizing branch lengths and ancestral states results in a contradiction.

In sufficient enough generality, we would look for any combination of $\{\tau_1,\tau_2\}$ that would result in this contradiction, but let's first focus on the two from the previous section.

\subsubsection{Classic formulations of topologies}

\begin{table}
\centering
\begin{tabular}{|l|l|}
    \hline
s   &$P(s)$\\
    \hline
0000&$1+x^2+y^2+4xy^2+x^2y^2$\\
0001&$1+x^2-y^2-x^2y^2$\\
0010&$1+x^2-y^2-x^2y^2$\\
0100&$1-x^2+y^2-x^2y^2$\\
1000&$1-x^2+y^2-x^2y^2$\\
0011&$1+x^2+y^2-4xy^2+x^2y^2$\\
0101&$1-x^2-y^2+x^2y^2$\\
1001&$1-x^2-y^2+x^2y^2$\\
    \hline
\end{tabular}    
\caption{Site pattern probabilities for Farris zone topology.
All values are multiplied by $1/8$.}
\end{table}

\begin{table}
\centering
\begin{tabular}{|l|l|}
    \hline
s   &$P(s)$\\
    \hline
0000&$1+2xy+2xy^2+x^2y+y^3+x^2y^2$\\
0001&$1+x^2y-y^3-x^2y^2$\\
0010&$1-x^2y+y^3-x^2y^2$\\
0100&$1+x^2y-y^3-x^2y^2$\\
1000&$1-x^2y+y^3-x^2y^2$\\
0011&$1+2xy-2xy^2-x^2y-y^3+x^2y^2$\\
0101&$1-2xy-2xy^2+x^2y+y^3+x^2y^2$\\
1001&$1-2xy+2xy^2-x^2y-y^3+x^2y^2$\\
    \hline
\end{tabular}    
\caption{Site pattern probabilities for Felsenstein zone topology.
All values are multiplied by $1/8$.}
\label{tab:sitepatprob-fels}
\end{table}

\begin{table}
\centering
\begin{tabular}{|l|ll|}
    \hline
    &00                              &01\\
    \hline
0000&$(1+x)^2(1+y)^{3*}$          &$(1+x)^2(1-y)^3$\\
0001&$(1+x)^2(1+y)^2(1-y)^*$       &$(1+x)^2(1+y)(1-y)^2$\\
0010&$(1+x)^2(1+y)^2(1-y)^*$       &$(1+x)^2(1+y)(1-y)^2$\\
0100&$(1+x)(1-x)(1+y)^{3*}$       &$(1+x)(1-x)(1-y)^3$\\
1000&$(1+x)(1-x)(1+y)^{3*}$       &$(1+x)(1-x)(1-y)^3$\\
0011&$(1+x)^2(1+y)(1-y)^{2}$          &$(1+x)^2(1-y)(1+y)^{2\ddagger}$\\
0101&$(1+x)(1-x)(1+y)^2(1-y)^{\dagger}$    &$(1+x)(1-x)(1+y)(1-y)^2$\\
1001&$(1+x)(1-x)(1+y)^2(1-y)^{\dagger}$    &$(1+x)(1-x)(1+y)(1-y)^2$\\
    \hline
    \hline
&10                           &11\\
    \hline
0000&$(1-x)^2(1+y)^2(1-y)$        &$(1-x)^2(1+y)(1-y)^2$\\
0001&$(1-x)^2(1+y)(1-y)^2$     &$(1-x)^2(1+y)^2(1-y)$\\
0010&$(1-x)^2(1+y)(1-y)^2$     &$(1-x)^2(1+y)^2(1-y)$\\
0100&$(1+x)(1-x)(1+y)^2(1-y)$     &$(1+x)(1-x)(1+y)(1-y)^2$\\
1000&$(1+x)(1-x)(1+y)^2(1-y)$     &$(1+x)(1-x)(1+y)(1-y)^2$\\
0011&$(1-x)^2(1-y)^3$       &$(1-x)^2(1+y)^{3\ddagger}$\\
0101&$(1+x)(1-x)(1+y)(1-y)^2$  &$(1+x)(1-x)(1+y)^2(1-y)^{\dagger}$\\
1001&$(1+x)(1-x)(1+y)(1-y)^2$  &$(1+x)(1-x)(1+y)^2(1-y)^{\dagger}$\\
    \hline
\end{tabular}    
\caption{Likelihood calculations for all site patterns and internal states of Farris topology.
Maxima determined row-wise (i.e., by site pattern).
Key: $^*$ unique maximum value corresponding to unique internal state; $^\dagger$ unique maximum value corresponding to multiple internal states; $^\ddagger$ multiple maximum values corresponding to multiple internal states.}
\end{table}

\begin{table}
\centering
\begin{tabular}{|l|ll|}
    \hline
    &00                              &01\\
    \hline
0000&$(1+x)^2(1+y)^{3*}$          &$(1+x)(1-x)(1+y)(1-y)^2$\\
0001&$(1+x)^2(1+y)^2(1-y)^{*}$       &$(1+x)(1-x)(1+y)^2(1-y)$\\
0010&$(1+x)(1-x)(1+y)^{3\ddagger}$       &$(1+x)^2(1+y)(1-y)^{2\ddagger}$\\
0100&$(1+x)^2(1+y)^2(1-y)^*$       &$(1+x)(1-x)(1-y)^3$\\
1000&$(1+x)(1-x)(1+y)^{3\ddagger}$       &$(1-x)^2(1+y)(1-y)^2$\\
0011&$(1+x)(1-x)(1+y)^2(1-y)^{\dagger}$          &$(1+x)^2(1+y)^2(1-y)$\\
0101&$(1+x)^2(1+y)(1-y)^{2\ddagger}$    &$(1+x)(1-x)(1+y)(1-y)^2$\\
1001&$(1+x)(1-x)(1+y)^2(1-y)^{\dagger}$    &$(1+x)(1-x)(1+y)(1-y)^2$\\
    \hline
    \hline
&10                           &11\\
    \hline
0000&$(1+x)(1-x)(1+y)(1-y)^2$        &$(1-x)^2(1+y)(1-y)^2$\\
0001&$(1+x)(1-x)(1-y)^3$     &$(1-x)^2(1+y)^2(1-y)$\\
0010&$(1-x)^2(1+y)(1-y)^2$     &$(1+x)(1-x)(1+y)(1-y)^2$\\
0100&$(1+x)(1-x)(1+y)^2(1-y)^{\ddagger}$     &$(1-x)^2(1+y)^2(1-y)$\\
1000&$(1+x)^2(1+y)(1-y)^{2\ddagger}$     &$(1+x)(1-x)(1+y)(1-y)^2$\\
0011&$(1-x)^2(1-y)^3$       &$(1+x)(1-x)(1+y)^2(1-y)^{\dagger}$\\
0101&$(1+x)(1-x)(1+y)(1-y)^2$  &$(1-x)^2(1+y)^{3\ddagger}$\\
1001&$(1+x)(1-x)(1+y)(1-y)^2$  &$(1+x)(1-x)(1+y)^2(1-y)^{\dagger}$\\
    \hline
\end{tabular}    
\caption{Likelihood calculations for all site patterns and internal states of Felsenstein topology.
Maxima determined row-wise (i.e., by site pattern).
Key: $^*$ unique maximum value corresponding to unique internal state; $^\dagger$ unique maximum value corresponding to multiple internal states; $^\ddagger$ multiple maximum values corresponding to multiple internal states.}
\label{tab:likelihoods-fels}
\end{table}


\begin{table}
\centering
\begin{tabular}{|l|l|}
    \hline
s   &$P(s)$\\
    \hline
0000&$1+x^2+z^2+2xyz+x^2y+z^2y+x^2z^2$\\
0001&$1+x^2-z^2+x^2y-z^2y-x^2z^2$\\
0010&$1-x^2+z^2-x^2y+z^2y-x^2z^2$\\
0100&$1+x^2-z^2+x^2y-z^2y-x^2z^2$\\
1000&$1-x^2+z^2-x^2y+z^2y-x^2z^2$\\
0011&$1-x^2-z^2-2xyz-x^2y-z^2y+x^2z^2$\\
0101&$1+x^2+z^2-2xyz+x^2y+z^2y+x^2z^2$\\
1001&$1-x^2-z^2-2xyz-x^2y-z^2y+x^2z^2$\\
    \hline
\end{tabular}    
\caption{Site pattern probabilities for Felsenstein zone topology.
All values are multiplied by $1/8$.}
\label{tab:sitepatprob-fels}
\end{table}

\begin{table}
\centering
\begin{tabular}{|l|ll|}
    \hline
    &00                              &01\\
    \hline
0000&$(1+x)^2(1+y)(1+z)^{2*}$          &$(1+x)(1-x)(1-y)(1+z)(1-z)$\\
0001&$(1+x)^2(1+y)(1+z)(1-z)^{\ddagger}$       &$(1+x)(1-x)(1-y)(1+z)^{2\ddagger}$\\
0010&$(1+x)(1-x)(1+y)(1+z)^{2\ddagger}$       &$(1+x)^2(1-y)(1+z)(1-z)^{\ddagger}$\\
0100&$(1+x)^2(1+y)(1+z)(1-z)^{\ddagger}$       &$(1+x)(1-x)(1-y)(1-z)^2$\\
1000&$(1+x)(1-x)(1+y)(1+z)^{2\ddagger}$       &$(1-x)^2(1-y)(1+z)(1-z)$\\
0011&$(1+x)(1-x)(1+y)(1+z)(1-z)^{\ddagger}$          &$(1+x)^2(1-y)(1+z)^{2\ddagger}$\\
0101&$(1+x)^2(1+y)(1-z)^{2\ddagger}$    &$(1+x)(1-x)(1-y)(1+z)(1-z)^{\ddagger}$\\
1001&$(1+x)(1-x)(1+y)(1+z)(1-z)^{\dagger}$    &$(1+x)(1-x)(1-y)(1+z)(1-z)$\\
    \hline
    \hline
&10                           &11\\
    \hline
0000&$(1+x)(1-x)(1-y)(1+z)(1-z)$        &$(1-x)^2(1+y)(1-z)^2$\\
0001&$(1+x)(1-x)(1-y)(1-z)^2$     &$(1-x)^2(1+y)(1+z)(1-z)$\\
0010&$(1-x)^2(1-y)(1+z)(1-z)$     &$(1+x)(1-x)(1+y)(1-z)^2$\\
0100&$(1+x)(1-x)(1-y)(1+z)^{2\ddagger}$     &$(1-x)^2(1+y)(1+z)(1-z)$\\
1000&$(1+x)^2(1-y)(1+z)(1-z)^{\ddagger}$     &$(1+x)(1-x)(1+y)(1-z)^2$\\
0011&$(1-x)^2(1-y)(1-z)^2$       &$(1+x)(1-x)(1+y)(1+z)(1-z)^{\ddagger}$\\
0101&$(1+x)(1-x)(1-y)(1+z)(1-z)^{\ddagger}$  &$(1-x)^2(1+y)(1+z)^{2\ddagger}$\\
1001&$(1+x)(1-x)(1-y)(1+z)(1-z)$  &$(1+x)(1-x)(1+y)(1+z)(1-z)^{\dagger}$\\
    \hline
\end{tabular}    
\caption{Likelihood calculations for all site patterns and internal states of Felsenstein topology.
Maxima determined row-wise (i.e., by site pattern).
Key: $^*$ unique maximum value corresponding to unique internal state; $^\dagger$ unique maximum value corresponding to multiple internal states; $^\ddagger$ multiple maximum values corresponding to multiple internal states.}
\label{tab:likelihoods-fels}
\end{table}

In the first case we are in a similar situation to the case of $x=z=0$ where $\tilde{y}>y$ implies $L(\tilde{y})>L(y)$.
The second case is more interesting.
We still require $\tilde{y}>y$, but we now have a further restriction.
We are interested in the simultaneous inequalities
$$
\frac{2x}{1+x^2} > \tilde{y} > y,
$$
$$
[7-2x^2+4x^2y-x^4]\log\frac{1+\tilde{y}}{1+y}+[1+2x^2-4x^2y+x^4]\log\frac{1-\tilde{y}}{1-y} > 0
$$
which can be written
$$
8\log\frac{1+\tilde{y}}{1+y}+[1+2x^2-4x^2y+x^4]\log\frac{(1-\tilde{y})(1+y)}{(1+\tilde{y})(1-y)} > 0.
$$
What can we say about this?

Now terms involving only $P(s|\tau,t)$ differ between $t^*$ and $t$ for the site patterns 0000 and 0101 but no others, allowing us to ignore them.
Additionally, all terms involving $P(\xi_s|s,\tau,t)$ will differ between $t^*$ and $t$ yielding a search for parameters such that
$$
\sum_{s\in\{0000,0101\}} P(s|\tau^*, t^*)\log P(s|\tau^*, t) + \sum_{s} P(s | \tau^*, t^*) \log P(\xi_s | s, \tau^*, t)
$$
is maximized at $t\neq t^*$.
Since $\log P(\xi_s|s,\tau,t)$ breaks up into a sum of terms involving $x$ and $y$ we have
$$
(6-4x^2-2x^4)\log(1+y)+\sum_{s\in\{0000,0101\}} P(s|\tau^*, t^*)[\log P(s|\tau^*, t)+\log P(\xi_s|s, \tau^*, t)].
$$
For a bit of clarity, let
$$
a(x,y) = (1+2x^2+4x^2y+x^4),
$$
$$
b(x,y) = (1+2x^2-4x^2y+x^4)
$$
so that we have
$$
(6-4x^2-2x^4)\log(1+y)+a(x,y)\log a(x,y)+b(x,y)\log b(x,y) + \sum_{s\in\{0000,0101\}} P(s|\tau^*, t^*)\log P(\xi_s|s, \tau^*, t).
$$
Canceling more terms shows the right-hand sum is
$$
(1+2x^2+4x^2y+x^4)\log(1+y) + 16x^2y\log(1+x) + P(0101|\tau^*, t^*)\log P(\xi_s|0101, \tau^*, t).
$$
That final right-hand term above is actually a component function since we are maximizing a likelihood, and the 0101 row in Table~\ref{tab:likelihoods} does not have a unique maximum value.
This value is either
$$
c_1(x,y) = (1+x)^4(1-y),
$$
$$
c_2(x,y) = (1+x)^2(1-x)^2(1+y).
$$
We are interested in two inequalities; if
$$
c_2(x,y) > c_1(x,y)
$$
then our whole objective function simplifies since many terms with $\log(1+y)$ cancel.
We can then focus on a second inequality, namely whether
$$
8\log(1+y) + x^2y(4\log(1+y)+16\log(1+x)) + a(x,y)\log a(x,y) + b(x,y)\log b(x,y) + c_2(x,y)...blah
$$
is greater than
$$
8\log(1+\tilde{y}) + x^2y(4\log(1+\tilde{y})+16\log(1+x)) + a(x,y) + b(x,y) + c_2(x,y)...
$$

\subsection{Dashed intuition}

I am having a tough time seeing the intuition behind this result.
If $x=z=0$ then we are flipping a coin for what happens at the internal node proceeding from the leaf.
Why would effectively constraining the interior nodes to be equal result in a larger value for the likelihood?
It doesn't have anything to do with efficiency, does it (i.e., by constraining the parameters to be equal we are using the same amount of data to estimate fewer parameters)?
Can we show that constraining any $\xi_i,\xi_j$ to be equal in $L_n$ will also result in problems?

I think what may help my sanity is proceeding with the above for the usual maximum likelihood story, that is, for
$$
\log L''' = \sum_{s} P(s | \tau^*, t^*) \cdot [\log P(s | \tau^*, t) + \log\sum_{\xi} P(\xi_s | s, \tau^*, t)]
$$
show
$$
L'''(\tau^*, t^*) > L'''(\tau^*, t)
$$
for $t\neq t^*$.
At least for $x=z=0$ we can see that these two quantities will be equal for all $y$ due to the symmetry of choosing the internal nodes.
Is this satisfying?

UPDATE: thinking about this more, it actually makes sense.
If we are maximizing over internal states, we will always prefer a branch length probability that will indicate no change occurs between them, effectively constraining all internal states to be the same given a particular site pattern.
As we can see in the table, the maximum value almost always occurs for internal states that are equivalent.
Seeking harmony between internal states will always increase the likelihood except in cases where branches to terminating leaves are sufficiently small.
So the question might be, now, whether or not we can prove a general result: since internal states seem to always maximize the likelihood by being constrained to be equal, what conditions are needed on the terminal branches (here $x$ and $z$) that will yield (in)consistent estimates?


\subsection{Plan of attack}

We recall \eqref{eq:inconsistency_inequality} and set $\tau^*=\tau_1$ and $\tau'=\tau_2$.
In other words, data are generated from the ``Farris zone'' topology, and we seek all values for $t^*$ where we can obtain a higher likelihood by proceeding with joint maximization assuming the ``Felsenstein zone'' topology.

\subsection{The cheater's approach}

Using \texttt{python}, we can numerically maximize \eqref{eq:site_pattern_profile_likelihood_mean} over $t$ separately for fixed topologies $\tau^*$ and $\tau'$.
For $t^*$, we can calculate these values over a grid of values for $(x^*, y^*)\in[0,1]^2$.
At worst, this will show us if we're on a fool's errand, but if not then we could benefit by gaining insight into what form a refinement of branch lengths yielding inconsistency would take.

